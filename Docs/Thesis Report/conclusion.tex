The IRS scheduler designs implemented in this thesis highlights a new approach in designing the scheduler for the IRS framework which involved by  moving the scheduler logic to kernel space. 
We introduced two different design approaches in moving the IRS scheduler to kernel space. 
The first approach triggers an IOCTL call to kernel space for every shared memory event.
In the second approach, we reduce the number of IOCTL calls made to the kernel space for every shared memory event. 
The reduction in the number of IOCTL calls is achieved by having a proxy checking for memory access permission during a shared memory event. 
The kernel space solutions presented with six different prototypes. 
The first four prototypes used the first approach and the last two prototypes were realized with the second approach. 
All the prototypes presented in this thesis were incremental improvements over its former.
We also introduced an alternative representation in the form of vector clocks for realizing the scheduling constraints enforced on threads in a multithreaded program. 
The vector clock representation was intended to reduce the code complexity and for overcoming one of the design challenges of this thesis. 

In our experiments, we compared the IRS user space and IRS kernel space solutions with the benchmarking programs - Fibonacci, Last Zero, Indexer and Dining Philosopher's Problem. 
In our experiments, we observed that IRS\_Sh performed poorly when the number of threads were more than the number of cores and when the memory constraints spanned all the threads in the benchmarking program. 
We also observed that the IRS\_Opt performed better in nearly all the benchmarks but, performed poorly when the memory constraints were spanned to fewer number of threads as observed with Indexer and Fibonacci experiments. 
Among the kernel space designs, prototypes 5 and 6 performed the best in nearly all the experiments. 
However, the experiments showed us that when the number of memory constraints came closer to the total number of shared memory events Prototypes 1 and 3 were better as it can be observed with Fibonacci and Last Zero benchmarks. 
In our experiments, we observed that kernel space designs perform better when number of memory constraints are higher and closer to the total number of shared memory events. 
However, the prototypes depicted in this thesis suffer from scalability problems as observed with the experiments done on scaling of the processor cores. 
The prototypes could be optimized by removing the C-structure representation of vector clock, which could help in overcoming a potential false sharing problem. 

An experimental study of all these prototypes suggested a new design outlook for moving the IRS scheduler to kernel space. 
Based on a preliminary analysis, it is evident that there is a need of solution which is a combination of IRS\_Opt and Prototypes 5-6. 
Such a design would require the custom yield and revive functionality from prototypes 5-6 and checking for memory access permission would be handled in the user space entirely. 
In short, the condition variable design used in IRS\_Opt would be replaced by the custom yield and revive functionality from Prototypes 5-6. 
Such an approach may help to further improve the performance of constrained execution of a multithreaded program on IRS environment.

