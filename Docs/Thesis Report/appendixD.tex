\begin{itemize}

\item 	API 	-   Application Programming Interface
\item	BDD		-	Binary Decision Diagrams
\item	CPU		-	Central Processing Unit
\item	DMT		-	Deterministic Multithreading
\item	IOCTL	-	Input Output Control
\item	IPC		-	Inter Process Communication
\item	IRS		-	Iterative Relaxed Scheduling
\item  {IRS\_Opt} - Another user space IRS solution discussed in section~ \ref{iter_rel_sched}. 
It uses an additional thread for handling the scheduling decisions and uses conditional variables to block a certain thread when memory access is restricted. 
\item {IRS\_Sh} - One of the IRS user space implementation discussed in section~\ref{iter_rel_sched}. 
It addresses the use of a busy waiting design for blocking a thread and use the other threads in the thread pool to signal the blocked thread. 
Thus, making the scheduling decision shared among the threads. 
This design does not have an additional thread for handling the scheduling decisions. 
\item	LKM		-	Loadable Kernel Module
\item	OS		-	Operating System
\item	POR		- 	Partial Order Reduction
\item	POSIX	- 	Portable Operating System Interface
\item	ProcFS  -   Process File System, a virtual file system in Linux Operating System.
\item {Proto\_1} - Section~\ref{sync_des} highlights all the prototypes used in this thesis to provide various solutions addressing the transition of scheduling decision to kernel space. 
Prototype 1 is the first synchronization discussed in section~\ref{sync_des} which uses a shared scheduler design similar to IRS\_Sh but, the blocking of threads is enforced by using semaphores in kernel space. 
\item {Proto\_2} - Prototype 2 is an extension of the previous prototype. 
Prototype also uses semaphores in kernel space for blocking a given thread but, the signaling the blocked thread is done by an additional thread which is similar to IRS\_Opt.
\item {Proto\_3} - Prototype 3 is a lot similar to Prototype 1 in the nature of behavior when it comes to design and its approach to scheduling. 
Main difference of Prototype 3 compared to Prototype 1 is the use of scheduler APIs instead of semaphores for blocking a given thread. 
\item {Proto\_4} - Prototype 4 is a design variant of Prototype 2. It uses the same scheduling approach as Prototype 2 but, differs in the blocking of a given thread. 
It uses scheduler APIs instead of semaphores. 
\item {Proto\_5} - Prototype 5 are an extension of Prototype 2, it is one of the designs which addresses the second approach discussed in section~\ref{sec_app}. 
As discussed in section~\ref{sec_app} it uses a proxy checking of memory access permission in user space.
\item {Proto\_6} - Prototype 6 are an extension of Prototype 4, it is another design which addresses the second approach discussed in section~\ref{sec_app}. 
As discussed in section~\ref{sec_app} it uses a proxy checking of memory access permission in user space.
\item Pthreads- POSIX Threads
\item RTID - Real Thread ID which is assigned within the proc file system for any thread created within the user land. 
\item SMP		- Symmetric Multi-Processor
\item StableMT 	- Stable Multithreading
\item TaskID 	- All threads are internally realized as tasks in kernel space and are allocated with an identifier which is task Identifier.
\item TLB	- Translation Lookaside Buffer
\item UTID 	- User defined thread ID which is relative inside the user program.
\item VM	-	Virtual machine
\end{itemize}
