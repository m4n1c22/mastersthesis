---background information related to thesis comes here---
\section{Software Verification}

Software programs are becoming increasingly complex. 
With the rise in complexity and technological advancements, components within a software have become susceptible to various erroneous conditions. 
Software verification have been perceived as a solution for the problems arising in the software development cycle. 
Software verification is primarily verifying if the specifications are met by the software. 

There are two fundamental approaches used in software verification - dynamic and static software verification. 
Dynamic software verification is performed in conjunction with the execution of the software. 
In this approach, the behavior of the execution program is checked- commonly known as Test phase. 
Verification is succeeding phase also known as Review phase. 
In dynamic verification, the verification adheres the concept of test and experimentation. 
The verification process handles the test and behavior of the program under different execution conditions. 
Static software verification is the complete opposite of the previous approach. 
The verification process is handled by checking the source code of the program before its execution. 
Static code analysis is one such technique which uses a similar approach. 

The verification of software can also be classified in perspective of automation - manual verification and automated verification. 
In manual verification, a reviewer manually verifies the software. 
Whereas in the latter approach, a script or a framework performs  verification. 

Software verification is a very broad area of research. 
This thesis work is focussed on automated software verification for concurrent programming models. 

\subsection{Automated Software Verification for Concurrent Programs}



\subsubsection{Partial Order Reduction}
\subsubsection{Lipton}
\subsubsection{Dynamic POR}

\section{Model Checking}
\section{Symbolic Execution}

\section{Iterative Relaxed Scheduling}
\section{Deterministic Multi-Threading}
