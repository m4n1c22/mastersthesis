\section{Software Verification}

Software programs are becoming increasingly complex. 
With the rise in complexity and technological advancements, components within a software have become susceptible to various erroneous conditions. 
Software verification have been perceived as a solution for the problems arising in the software development cycle. 
Software verification is primarily verifying if the specifications are met by the software\cite{ghezzi2002fundamentals}.

There are two fundamental approaches used in software verification - dynamic and static software verification\cite{ghezzi2002fundamentals}. 
Dynamic software verification is performed in conjunction with the execution of the software. 
In this approach, the behavior of the execution program is checked- commonly known as Test phase. 
Verification is succeeding phase also known as Review phase. 
In dynamic verification, the verification adheres the concept of test and experimentation. 
The verification process handles the test and behavior of the program under different execution conditions. 
Static software verification is the complete opposite of the previous approach. 
The verification process is handled by checking the source code of the program before its execution. 
Static code analysis is one such technique which uses a similar approach. 

The verification of software can also be classified in perspective of automation - manual verification and automated verification. 
In manual verification, a reviewer manually verifies the software. 
Whereas in the latter approach, a script or a framework performs  verification. 

Software verification is a very broad area of research. 
This thesis work is focussed on automated software verification for multithreaded programming. 

\section{Multithreaded Programming}

Computing power has grown over the years. 
Advancements are made in the domain of computer architecture by moving the computing power from single-core to multi-core architecture. 
With such advancement, there were needs to adapt the programming designs from a serialized execution to more parallelizable execution. 
Various parallel programming models were perceived to accommodate the perceived progression. 
Multithreaded programming model was one of the designs considered for the performance boost in computing. 

Threads are a small tasks executed by a scheduler on an operating system, where the resources such as the processor, TLB (Translation Lookaside Buffer), cache, etc., are shared between them. 
Threads share the same address space and resources. 
Multithreading addresses the concept of using multiple threads for having concurrent execution of a program on a single or multi-core architectures. 
Inter-thread communication is achieved by shared memory. 
Mapping the threads to the processor core is done by the operating system scheduler. 
Multithreading is only supported in operating systems which has multitasking feature. 

Advantages of using multithreading include: 
\begin{itemize}
\item	Fast Execution
\item	Better system utilization
\item	Simplified sharing  and communication
\item 	Improved responsiveness
\item	Parallelization
\end{itemize}

Disadvantages:
\begin{itemize}
\item	Race conditions
\item	Deadlocks with improper use of locks/synchronization
\item	Cache misses when sharing memory
\end{itemize}


\subsubsection{Partial Order Reduction}
\subsubsection{Lipton}
\subsubsection{Dynamic POR}

\section{Model Checking}
\section{Symbolic Execution}

\section{Iterative Relaxed Scheduling}
\section{Deterministic Multi-Threading}
