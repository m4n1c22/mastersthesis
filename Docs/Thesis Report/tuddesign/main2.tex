%======================================================
% Technische Universitaet Darmstadt
% Fachbereich Elektrotechnik und Informationstechnik
% Fachbereich Informatik (Zweitmitglied)
% Fachgebiet Multimedia Kommunikation (KOM)
% Prof. Dr.-Ing. Ralf Steinmetz
%======================================================
% Template for Theses
% VERSION 1.0 (October 2009)
% Use pdfLaTeX (other possible, but not supported)
% Contact at KOM: Andr\'e Miede (andre.miede@kom...)
%======================================================
% Official TUD-LaTeX-files have to be installed:
% http://exp1.fkp.physik.tu-darmstadt.de/tuddesign/
% Refer to the manuals and forum for details
%======================================================
\documentclass[longdoc,accentcolor=tud1b,12pt,paper=a4]{tudreport}
%======================================================
% colorback = Bereich unter Titel mit Hintergrundfarbe
% colorbacktitle = Titel mit Hintergrundfarbe (Akzent)
% KOM-Blau = accentcolor=tud1b		
% Grau = accentcolor=tud0a 
% blackrule fuer schwarze Leiste
% nochapterpage = do not start chapters on new page
% oneside = print only on one side of the page
%======================================================

%======================================================
% General package loading and definitions
%======================================================
\usepackage{inputenc} 
\usepackage{textcomp} 
\usepackage{ngerman}
\usepackage[american,ngerman]{babel}
\usepackage{xspace}
\usepackage[fleqn]{amsmath} % math environments and more by the AMS 
\newcounter{dummy} % necessary for correct hyperlinks (to index, bib, etc.)
\newcommand{\myfloatalign}{\centering} % how all the floats will be aligned

%======================================================
% DEEDS-modifications of the TUD-layout
%======================================================
% reduce font size of page footers and headers (fancyhdr)
\renewcommand{\footerfont}{\fontfamily{\sfdefault}\fontseries{m}\fontshape{n}\footnotesize\selectfont}
% remove space between items 
\usepackage{enumitem}
	\setenumerate{noitemsep}
	\setitemize{noitemsep}
	\setdescription{noitemsep}
\setlist{nolistsep}

%======================================================
% Package loading for example contents (content.tex)
%======================================================
\usepackage{tabularx} % better tables
\setlength{\extrarowheight}{3pt} % increase table row height
\usepackage{booktabs}
\usepackage{caption}
\captionsetup{format=hang,font=small}
\usepackage[square,numbers]{natbib}
\usepackage{subfig}
\usepackage[stable,bottom]{footmisc}
\usepackage{listings}
\lstset{basicstyle=\ttfamily,
  showstringspaces=false,
  commentstyle=\color{red},
  keywordstyle=\color{blue}
}

%======================================================
% Important information: to be set here and only here
%======================================================
\newcommand{\deedsTitle}{Realizing Iterative-Relaxed Scheduler in Kernel Space\xspace}
\newcommand{\deedsThesisType}{Master-Arbeit\xspace} % Diplomarbeit Studienarbeit Master-Arbeit Bachelor-Arbeit
\newcommand{\deedsID}{2662284\xspace}
\newcommand{\deedsName}{Sreeram Sadasivam\xspace}
\newcommand{\deedsSubmissionDate}{16. Marz 2018\xspace}% use only this date format
\newcommand{\deedsGutachter}{Gutachter: Prof. Neeraj Suri Ph.D\xspace}
\newcommand{\deedsBetreuer}{Betreuer: Patrick Metzler\xspace}
%\newcommand{\deedsExternerBetreuer}{Externer Betreuer: Dr. Nokia Siemens\xspace}

%======================================================
% Setup for hyperref
%======================================================
\usepackage[pdftex,hyperfootnotes=true,pdfpagelabels]{hyperref}
	\pdfcompresslevel=9
	\pdfadjustspacing=1 
\hypersetup{%
    colorlinks=false, linktocpage=false, pdfstartpage=1, pdfstartview=FitV,%
    breaklinks=true, pdfpagemode=UseNone, pageanchor=true, pdfpagemode=UseOutlines,%
    plainpages=false, bookmarksnumbered, bookmarksopen=true, bookmarksopenlevel=1,%
    hypertexnames=true, pdfhighlight=/O, %nesting=true,%frenchlinks,%
    %urlcolor=tud1b, linkcolor=tud1b, citecolor=tudtud1bccent,
    pdftitle={\deedsTitle, \deedsThesisType, \deedsID},%
    pdfauthor={\deedsName, DEEDS, TU Darmstadt},%
    pdfsubject={},%
    pdfkeywords={},%
    pdfcreator={},%
    pdfproducer={}%
}

%============================================
% Setup of the title page (do not change)
%============================================
\title{\deedsTitle}
\subtitle{\deedsThesisType}
\subsubtitle{\deedsName \\ \deedsID}
%\setinstitutionlogo[height]{deeds_info}
\institution{\raggedleft Fachbereich Informatik\\[\baselineskip]%
	Fachgebiet Dependable Embedded Systems and Software \\%(DEEDS)
	Prof. Neeraj Suri Ph.D%
}

%============================================
% Setup of the title backside (do not change)
%============================================
\lowertitleback{%
	Technische Universit\"at Darmstadt \\%
	Fachbereich Informatik \\[\baselineskip]%
        Fachgebiet Dependable Embedded Systems and Software \\%(DEEDS)
	Prof. Neeraj Suri Ph.D% Fachgebiet Dependable Embedded Systems and Software \\%(DEEDS)
}

\uppertitleback{%
	\deedsTitle \\%
	\deedsThesisType \\%
	\deedsID \\[\baselineskip]%
	Eingereicht von \deedsName \\%
	Tag der Einreichung: \deedsSubmissionDate \\[\baselineskip]%
	\deedsGutachter \\%
	\deedsBetreuer \\%
	%\deedsExternerBetreuer%
}



%=====================================================
% CUSTOM SETUP STUFF
%=====================================================
\usepackage{tikz}
\usetikzlibrary{shadows,arrows}
\usetikzlibrary{shapes.geometric,arrows}
%\tikzstyle{line}=[draw]


%\usepackage{program}
\usepackage{algorithmicx}
\usepackage{algpseudocode}
% Define the layers to draw the diagram
\pgfdeclarelayer{background}
\pgfdeclarelayer{foreground}
\pgfsetlayers{background,main,foreground}
 
% Define block styles  
\tikzstyle{materia}=[draw, fill=white!20, text width=7.0em, text centered,
  minimum height=1.5em,drop shadow]
\tikzstyle{practica} = [materia, text width=8em, minimum width=10em,
  minimum height=3em, rounded corners, drop shadow]
\tikzstyle{texto} = [above, text width=6em, text centered]
\tikzstyle{linepart} = [draw, thick, color=black!50, -latex', dashed]
\tikzstyle{line} = [draw, thick, color=black!50, -latex']
\tikzstyle{ur}=[draw, text centered, minimum height=0.01em]
 
% Define distances for bordering
\newcommand{\blockdist}{1.3}
\newcommand{\edgedist}{1.5}

\newcommand{\practica}[2]{node (#1) [practica]
  {Pr\'actica #1\\{\scriptsize\textit{#2}}}}


\newcommand{\spacelayer}[3]{node (#1) [practica]
  {#2\\{\scriptsize\textit{#3}}}}


% Draw background
\newcommand{\background}[5]{%
  \begin{pgfonlayer}{background}
    % Left-top corner of the background rectangle
    \path (#1.west |- #2.north)+(-0.5,0.5) node (a1) {};
    % Right-bottom corner of the background rectanle
    \path (#3.east |- #4.south)+(+0.5,-0.25) node (a2) {};
    % Draw the background
    \path[fill=gray!10,rounded corners, draw=black!50, dashed]
      (a1) rectangle (a2);
    \path (a1.east |- a1.south)+(0.8,-0.3) node (u1)[texto]
      {\scriptsize\textit{#5}};
  \end{pgfonlayer}}

\newcommand{\transreceptor}[3]{%
  \path [linepart] (#1.east) -- node [above]
    {\scriptsize Transreceptor #2} (#3);}



\usepackage{listings}
\usepackage{color}
 
\definecolor{codegreen}{rgb}{0,0.6,0}
\definecolor{codegray}{rgb}{0.5,0.5,0.5}
\definecolor{codepurple}{rgb}{0.58,0,0.82}
\definecolor{backcolour}{rgb}{0.95,0.95,0.92}
 
\lstdefinestyle{mystyle}{
    backgroundcolor=\color{backcolour},   
    commentstyle=\color{codegreen},
    keywordstyle=\color{magenta},
    numberstyle=\tiny\color{codegray},
    stringstyle=\color{codepurple},
    basicstyle=\footnotesize,
    breakatwhitespace=false,         
    breaklines=true,                 
    captionpos=b,                    
    keepspaces=true,                 
    numbers=left,                    
    numbersep=5pt,                  
    showspaces=false,                
    showstringspaces=false,
    showtabs=false,                  
    tabsize=2
}
 
\lstdefinestyle{customset}{   
	backgroundcolor=\color{white},   
    commentstyle=\color{codegreen},
    keywordstyle=\color{magenta},
    stringstyle=\color{codepurple},
    basicstyle=\footnotesize,
    breakatwhitespace=false,         
    breaklines=true,                 
    captionpos=b,                    
    keepspaces=true,    
    showspaces=false,                
    showstringspaces=false,
    showtabs=false,                  
    tabsize=1,
    frame=single,
    numbers=none,
    language=C
}


\lstdefinestyle{customc}{  
  belowcaptionskip=1\baselineskip,
  breaklines=true,
  frame=single,
 % xleftmargin=\parindent,
  language=C,
  showstringspaces=false,
  basicstyle=\footnotesize\ttfamily,
  keywordstyle=\bfseries\color{red!40!black},
  commentstyle=\itshape\color{gray!40!black},
  identifierstyle=\color{black},
  stringstyle=\color{orange},
  morekeywords={thread, thread_init, thread_exec, join, main},
}

	
%======================================================
% MAIN DOCUMENT STARTS HERE
%======================================================
\begin{document}
%======================================================
	% The front matter
	%======================================================
	\pagenumbering{roman}
	\frenchspacing
	\raggedbottom
	\selectlanguage{american} % american ngerman
	\maketitle
	
	\chapter*{Ehrenw\"ortliche Erkl\"arung}
	Hiermit versichere ich, die vorliegende \deedsThesisType ohne Hilfe Dritter und nur mit den angegebenen Quellen
    und Hilfsmitteln angefertigt zu haben. Alle Stellen, die aus den Quellen entnommen wurden, sind als solche
    kenntlich gemacht worden. Diese Arbeit hat in dieser oder \"ahnlicher Form noch keiner Pr\"ufungsbeh\"orde vorgelegen.
    Die schriftliche Fassung stimmt mit der elektronischen Fassung \"uberein.
    
	
	\vspace{1.5cm}
	
	\noindent Darmstadt, den \deedsSubmissionDate\hfill \deedsName
	
	\tableofcontents
	\listoffigures
	\listoftables

	\begin{abstract}
	Abstract comes here...
	\end{abstract}		
	
	
	%======================================================
	% The main matter (insert your contents here)
	%======================================================
	\cleardoublepage
	%\graphicspath{ {./gfx/} }
	\pagenumbering{arabic}
	\chapter{Background \label{bkgd}}
	---background information related to thesis comes here---

	
	%\chapter{Related Work \label{rel_work}}
	%The previous chapter highlighted the need for debugging tools or methodologies for solving concurrency bugs in multithreaded programming. 
This thesis is conceived with a solution based on iterative relaxed scheduling(IRS). 
However, there are other techniques which address various solutions using deterministic multithreading(DMT). 
In this chapter, we explore various DMT based solutions and the similarities they share with IRS. 

\section{DTHREADS}

\citet{dthreads} presents a deterministic multithreading runtime system. 
It is build on top of PThreads(POSIX Thread) library in Linux. 
The dthreads library replaces most of the pthread library functions with its own implementation enforcing determinism in execution of threads. 
All the threads created using dthreads library are created as processes and they use a deterministic memory commit protocol for synchronizing the conceived shared memory state. 
The idea of using ``threads-as-processes'' paradigm is motivated from the work of \citet{grace}. 
By moving the design to processes, dthreads eliminates false sharing and provides protection faults. 
For simplicity we would call these threads as dthreads. 
``Twining and diffing'' technique is used to perform the deterministic  memory commit protocol in dthreads. 
Dthreads are run independently until they reach a synchronization point where the diffing step of the above technique takes place. 
It compares the modification in the memory page with the twin page with contains the shared state. 
Each dthread enters the differential comparison(diffing) step based on token being passed around the dthreads. 
Dthreads consists of two phases of execution: parallel and serial phases. 
The twining-diffing step occurs in the serial phase of execution. 
Transitions to serial phase occurs statically. 
Any synchronization operation will result in a transition to serial phase. 

Dthreads creates private, per-process copies of modified pages between commits. 
This would increase the program's memory footprint by the number of modified pages between synchronization operations. 
In case of IRS, we have do not change the implementation of pthreads. 
Therefore, the above problem never occurs in IRS however, there is a possibility of false sharing. 
IRS implementation emphasizes on memory level granularity whereas, dthreads focuses on the synchronization operations encountered in the user program. 
IRS follows a recorder-replay model whereas, dthreads uses ``Twining and diffing'' of threads. 
In IRS, we have a verifier which records the execution traces that are deemed to be safe considering the correctness of the program execution. 
It also consists of a scheduler which can be considered as the replay part of the model. 
In this thesis, we migrate the scheduler module to the kernel space from user space. 
The scheduler would run the instrumented user program with the execution traces generated by the verifier. 
Dthreads does not have a verifier or a separate scheduler module instead, it has library level support to enforce the determinism in execution. 
Also it does not instrument the user program. 
Dthreads is C/C++ library support implemented entirely in user space whereas, in this thesis we highlight the IRS scheduler implemented in kernel space. 

\section{GRACE}

\citet{grace} presents a deterministic library support in Grace. 
The design is similar to the Dthreads implementation. 
Grace also uses ``threads-as-processes'' paradigm. 
However, it is primarily targeted at fork-join models. 
Grace focuses on multithreaded designs which highlight thread creation and joining. 
The reason for the inclusion is that it also falls under the category of DMT based solutions.
However, it has lot drawbacks - it focuses on fork-join models only. 
It is similar to dthreads in a lot of regard.  
IRS is completely different to the Grace implementation.

\section{PEREGRINE}

\citet{peregrine} conceives an alternative DMT solution with schedule relaxation in PEREGRINE runtime system. 
It is a record-replay based implementation. 
It combines two different scheduling designs - sync schedule and memory schedules. 
The hybrid scheduling design is enforce efficiency and determinism in the execution of multithreaded program. 
Unlike the previously mentioned DMT solutions, PEREGRINE uses an instrumentor in LLVM and a user space scheduler for the replay of execution trace. 
PEREGRINE executes the multithreaded program with a certain input to generate its execution trace. 
The recorder records the trace for the given input of the program. 
Replayer/scheduler reuses the same execution trace for the given input of the program. 
It enforces the execution trace on the user program for same input thus, providing a level of determinism in its execution. 
It shares a lot of similarities with IRS. 
IRS is also record-replay based design. 
Both these designs provide memory level granularity. 
However, IRS design generates more traces with less memory level constraints for iteration therefore, retaining some level of non-determinism in the execution of the program. 
The memory level determinism in IRS is enforced based on the order of the memory access. Whereas, in case of PEREGRINE it is enforced based on the output of the program. 
PEREGRINE uses the same execution trace for different inputs to the multithreaded program. 
Whereas, IRS improves the degree of non-determinism in the execution of multithreaded program with every iteration of verifier.

\section{KENDO}

KENDO is another DMT solution proposed by \citet{kendo}, which uses modified Linux kernel to support deterministic logical time. 
KENDO is a software framework, which enforces weak deterministic execution of general purpose lock-based C/C++ based multithreaded programs.  
Weak determinism ensures a deterministic order of all lock acquisitions for a given program input.   
KENDO is a subset of pthreads library. 
It achieves determinism with the use of deterministic logical time, which is used to track the progress of each thread in a deterministic manner. 
KENDO has a kernel level implementation to enforce deterministic execution of threads. 
The IRS implementation highlighted in this thesis focuses on a scheduler implemented in kernel space for enforcing the memory constraints provided in the execution traces by the verifier. 
KENDO does not have any instrumentation of user program unlike PEREGRINE or IRS. 
KENDO only focuses on determinism in lock acquisitions and not on all shared memory accesses whereas, IRS addresses memory-level granularity for all shared memory accesses in the multithreaded program.

\section{COREDET}

\citet{coredet} came up with a compiler and runtime system enforcing deterministic multithreaded execution in COREDET. 
It is another runtime implementation based on DMT. 
COREDET has two phases - parallel and serial phases similar to the Dthreads. 
It has an instrumentor tool in LLVM for instrumenting memory events similar to PEREGRINE. 
COREDET is one of the first DMT solution which addressed the shared memory events and provided memory level granularity. 
COREDET can be executed in two different ways - ownership tracking and store buffering. 
First approach tracks the ownership of data and serializes the execution whenever threads communicate. Such an approach yields sequentially consistent executions and lower overheads, but lower scalability. 
Second approach uses memory versioning without any form of speculation and relaxes memory ordering, yielding higher scalability at the cost of higher overheads.  
It shares a lot of similarities with IRS implementation. 
Both the solutions use LLVM for instrumenting memory events in the multithreaded program. 
However, COREDET uses a round-robin scheduling when it enters a serial phase of execution. 
Whereas in case of IRS on occurrence of a memory event, the scheduler checks for the memory access permission for the given thread with the recorded trace. 
COREDET does not have any record-replay implementation. 
It can be conceived as a runtime implementation with emphasis on fine grained memory access with serialized commits.  


\section{PARROT}

PARROT is another runtime solution based on DMT from \citet{parrot}. 
Compared to other DMT solutions which maps one schedule for one input as depicted in figure~\ref{determinisitic_mapping}, PARROT proposes an approach which uses stable multithreading(StableMT). 
In StableMT, we reuse each schedule on a wide range of inputs, mapping all inputs to a dramatically reduced set of schedules. 
PARROT is a pthread compactible implementation. 
PARROT provides weak determinism similar to KENDO but offers stability. 
PARROT can be integrated with DBUG\citep{dbug} - open source model checker in Linux for determining bugs in the schedules. 
\citet{parrot} shows us that PARROT-DBUG ecosystem is more effective than either system alone. 
DBUG checks the schedules that matter to PARROT and the developers. 
Whereas, PARROT reduces the number of schedules to be checked by DBUG thus, increasing the coverage of DBUG. 
Compared to IRS, PARROT does not have any static code analysis done inside the multithreaded programs. 
The determinism provided by PARROT is relative to three factors: external input, performance critical sections, data races with respect to the enforced synchronization schedules. 
IRS focuses on memory level granularity whereas, PARROT is focused on weak determinism similar to KENDO. 
PARROT-DBUG focuses on StableMT with exhaustive testing of all schedules whereas in IRS, the execution of a multithreaded program can be initiated with a single execution trace from the verifier. 
IRS generates a new trace for every iteration whereas in case of PARROT-DBUG, the execution is blocked until the DBUG checks all the schedules.

\section*{Inference}

From the above sections, it is abundantly clear that there not any solutions which come close to the IRS. 
COREDET is the only implementation which seems to provide memory level granularity similar to IRS. 
PEREGRINE is another implementation which depicts a record-replay paradigm similar to IRS. 
PARROT-DBUG presents a StableMT focused on checking a set of reduced schedules for all the provided inputs in the multithreaded program. 
Other solutions presented in this chapter focus on DMT solutions aimed at synchronization operations rather than memory level accesses. 


	
	%\chapter{Design Challenges}
	%\input{challenges}	
	
	%\chapter{Designs}
	%\section{Design with no checking in user space}

In the following designs, we address the use of check permission of memory access method entirely in Kernel space.

\subsection{Design with no additional scheduler thread} \label{no_check_no_add}

The design described in this section addresses the use of no additional scheduler thread. 


\subsubsection*{Trace Registration}

\begin{tikzpicture}[scale=0.7,transform shape]
 
  % Draw diagram elements
  % trace registration area
  \path \spacelayer {TRACEFILE}{Trace File}{Input to the scheduler};  
  \path (TRACEFILE.south)+(0.0,-2.0)\spacelayer {TRACEOBJ}{Trace Object}{Trace object is created and communicated to kernel space};  
  \path (TRACEOBJ.east)+(5.0,0.0) \spacelayer {TRACEPROCFS}{Trace Reg File}{Proc FS custom file for passing the trace object to kernel space};
  \path (TRACEPROCFS.east)+(4.5,0.0) \spacelayer {TRACESYSCALL}{SYSCALL\\ TRACE}{write() system call};  
  \path (TRACESYSCALL.south)+(0.0,-3.0) \spacelayer {TRACECALLBACK}{Trace Write Callback}{Callback function triggered inside LKM during write of trace object};
  
  % Draw arrows between elements

  %thread registration block
  \path [line] (TRACEFILE.south) -- node [left] {$trace$}
									 node [right] {$file$} (TRACEOBJ);
  \path [line] (TRACEOBJ.east) -- node [above] {$trace_reg()$} (TRACEPROCFS);

  \path [line] (TRACEPROCFS.east) -- node [above] {$write()$} (TRACESYSCALL);
  
  \path [linepart] (TRACESYSCALL.south) -- node [left] {$callback$}
                                 node [right]{$triggered$} (TRACECALLBACK);  
  
  %background generation block
  \background{TRACEFILE}{TRACEFILE}{TRACEOBJ}{TRACEOBJ}{User Space}
  \background{TRACESYSCALL}{TRACESYSCALL}{TRACECALLBACK}{TRACECALLBACK}{Kernel Space}

\end{tikzpicture}

The trace file is passed on as an input for the scheduler. 
In the above flow diagram, the trace file is read by the main user thread at the start of its execution. 
It parses the file, creates and passes the trace object to the kernel space as string via a custom file created in the proc file system. 

\subsubsection*{Thread Registration}

\begin{tikzpicture}[scale=0.7,transform shape] 
  
  % thread registration block area
  \path \spacelayer {THREADINIT}{Thread\\ Created}{thread\_init()};  
  \path (THREADINIT.south)+(0.0,-2.0)\spacelayer {USERTHREADREG}{Thread\\ Registration}{reg\_thread()};  
  \path (USERTHREADREG.east)+(5.0,0.0) \spacelayer {REGPROCFS}{Registration File}{Proc FS custom file for communication};
  \path (REGPROCFS.east)+(4.5,0.0) \spacelayer {REGSYSCALL}{SYSCALL\\ THREAD\\ REGISTRATION}{write() system call};
  \path (REGSYSCALL.south)+(0.0,-2.0) \spacelayer {REGCALLBACK}{Write Callback}{Callback function triggered inside LKM during write of thread ID};

  \background{THREADINIT}{THREADINIT}{USERTHREADREG}{USERTHREADREG}{User Space}
  \background{REGSYSCALL}{REGSYSCALL}{REGCALLBACK}{REGCALLBACK}{Kernel Space}
  
   %Draw arrows between elements
   %thread registration block
  \path [line] (THREADINIT.south) -- node [left] {$register$}
									 node [right] {$thread$} (USERTHREADREG);
  \path [line] (USERTHREADREG.east) -- node [above] {$reg_thread()$} (REGPROCFS);

  \path [line] (REGPROCFS.east) -- node [above] {$write()$} (REGSYSCALL);
  
  \path [linepart] (REGSYSCALL.south) -- node [left] {$callback$}
                                 node [right]{$triggered$} (REGCALLBACK);

\end{tikzpicture}

In the above picture, the registration block happens when a user thread is created. 
The registration happens via a custom proc file system. 




\subsubsection*{Memory Assessment}

Prior to any global memory access, the given design would invoke \emph{IOCTL} command with \emph{CTXT\_SWITCH} and thread id of the thread which addressed the memory event as its parameters. 


\begin{tikzpicture}[scale=0.7,transform shape]
 
 
  % memory assessment area
  
  \path \spacelayer {BEFOREMA}{Before M.A}{A callback is triggered before memory access is made to the global memory};
  \path (BEFOREMA.south)+(0.0,-2.0) \spacelayer {REQCTXTSWITCH}{Request \\ Context Switch}{ioctl(CTXT\_SWITCH, tid)};
  
  \path (REQCTXTSWITCH.south)+(0.0,-2.0) \spacelayer {MEMACCESS}{Memory\\ Access}{The Actual global Memory Access by the thread};

  \path (MEMACCESS.south)+(0.0,-2.0) \spacelayer {AFTERMA}{After M.A}{A callback is triggered after memory access is made to the global memory};
  \path (AFTERMA.south)+(0.0,-3.0) \spacelayer {USRSIGOTHERS}{Request \\ Signal Other Threads}{ioctl(SIG\_OTHERS, tid)};  
  
  
  \path (REQCTXTSWITCH.east)+(8.0,2.0) \spacelayer {CONTEXTSWITCH}{Context Switch call}{If memory access is restricted, signal\_other\_threads() and call down(thread\_sem[tid]) }; 
  \path (CONTEXTSWITCH.east)+(3.0,-5.0) \spacelayer {CHECKTRACE}{Check trace}{Checks if the execution in the input trace is valid or not.}; 
  \path (CONTEXTSWITCH.west)+(-2.0,-6.0) \spacelayer {THREADSEM}{THREAD\\ Semaphore}{Array of semaphores used by corresponding thread. Up and Down calls are made.}; 
 \path (CONTEXTSWITCH.south)+(0.0,-10.0) \spacelayer {SIGNALOTHERS}{Signal other\\ Threads}{Signals all the permitted threads to be resumed by calling up() on their respective semaphores. For verifying the permission internal call is made to checktrace}; 
 
  
  

  % Draw arrows between elements

 
 
  

  %memory assessment block
  \path [line] (BEFOREMA.south) -- node [left] {$req$} (REQCTXTSWITCH);
  \path [line] (AFTERMA.south) -- node [left] {$req$} (USRSIGOTHERS);

  \path [line] (REQCTXTSWITCH.east) -- node [midway,above,sloped] {$ioctl()$} (CONTEXTSWITCH);
  
  \path [line] (CONTEXTSWITCH.south) -- node [left] {} (CHECKTRACE);
  \path [line] (CONTEXTSWITCH.south) -- node [midway,above,sloped] {down(sem[tid])} (THREADSEM);
  \path [line] (CONTEXTSWITCH.south) -- node [left] {} (SIGNALOTHERS);
  \path [line] (CHECKTRACE.north) -- node [midway,above,sloped] {allowed\ or\ restricted} (CONTEXTSWITCH);  
   
  \path [line] (SIGNALOTHERS.north) -- node [left] {} (CHECKTRACE);   
  \path [line] (CHECKTRACE.south) -- node [midway,above,sloped] {allowed\ or\ restricted} (SIGNALOTHERS);
  \path [line] (SIGNALOTHERS.north) -- node [midway,above,sloped] {up(sem[tid])} (THREADSEM);
   
   \path [line] (USRSIGOTHERS.east) -- node [midway,above,sloped] {$ioctl()$} (SIGNALOTHERS);
  %\path [linepart] (CONTEXTSWITCH.west) -- node [above] {$thread$}
                                 %node [right]{$resume$} (THREADRESUME);

  %background generation block
 
  \background{BEFOREMA}{BEFOREMA}{USRSIGOTHERS}{USRSIGOTHERS}{User Space}
  \background{THREADSEM}{CONTEXTSWITCH}{CHECKTRACE}{SIGNALOTHERS}{Kernel Space}
  

\end{tikzpicture}


\newpage
\subsubsection*{Pseudo Implementation}
\begin{lstlisting}[title=Data Types Section used by user space and kernel space, style=customc]
enum IOCTL CMDS  { 
	GET_CURR_CLK = 1, 
  	CTXT_SWITCH = 2, 
  	SIGNAL_OTHER_THREADS = 3,
  	RESET_CLK = 4,
  	SET_MY_CLK = 5
}
enum  mem_access{
	e_ma_restricted = 0,
	e_ma_allowed 	= 1
} 
struct vec_clk {
	int clocks[THREAD_COUNT],
}
struct trace_node {
	thread_id_t tid;
	vec_clk clk;
	int valid;
}

struct trace {
	trace_nodes trace_obj_arr[TRACE_LIMIT];
}
\end{lstlisting}

\begin{lstlisting}[title=Check Permission for memory access, style=customc]
mem_access check_mem_acc_perm(vec_clk* curr_vec_clk, vec_clk* trace_inst, thread_id_t tid) {

   int i;
   if(trace_inst->clocks[tid-1] == curr_vec_clk->clocks[tid-1]) 
   {
     for i in range(0, THREAD_COUNT) 
     {
	if(i!=(tid-1)) 
	{
	 if(trace_inst->clocks[i] <= curr_vec_clk->clocks[i]) 
	 {
	 	continue;
	 }
	 else 
	 {
	 	return e_ma_restricted;
	 }
	}
     }
   }
   else if(trace_inst->clocks[tid-1] < curr_vec_clk->clocks[tid-1]) 
   {
   	return e_ma_restricted;
   }
   return e_ma_allowed;
}

\end{lstlisting}


\begin{lstlisting}[title=User Space Implementation, style=customc]
BeforeMA() {	
	ioctl(CTXT_SWITCH, thread_id);	
}

AfterMA() {	
	ioctl(SIGNAL_OTHER_THREADS, thread_id);
}

reset_clock() {
	ioctl(RESET_CLK);
}

//This method is defined by the thread library which is used by the user
thread_create_impl(thread t) {
	t ->thread_init(tid);
	t ->thread_exec(thread_function);
}

thread_function() {
	reg_thread();	//This method increments a threadcount variable in kernel space.
	....	
	Before_MA(); 	//function triggered before accessing the global memory
	Mem_Access();   //global memory access permitted for the thread
	AfterMA();		//function triggered after accessing the global memory		
	....
	thread_exit()
}

trace_reg() {	
	fd = open("/proc/trace_reg",O_RDWR);
	close(fd);	
}

main() {	
	trace_reg()
	thread t = thread_create(tid, thread_function); 
	//thread_create_impl() is called internally
	.....
	t.join();	
	return EXIT_SUCCESS;
}


\end{lstlisting}
\begin{lstlisting}[title=Kernel Space - General module definitions, style=customc]
semaphore threads_sems[THREAD_COUNT];
int wait_queue[THREAD_COUNT];
trace trace_obj;
vec_clk curr_clk;
int thread_count = 0;

module_init() {
	for i in range(0,THREAD_COUNT) {
		init(threads_sem[i] = 0;
		wait_queue[i] = 0;
		curr_clk[i] = 0;
	}
	alloc_ioctl_device();//method used to allocate ioctl device.
}

trace_reg_callback() {
	//The method parses the trace which is passed as string and stores in trace_obj
}

reg_thread_callback() {
	thread_count++;
}

\end{lstlisting}
\newpage
\begin{lstlisting}[title=Kernel Space - IOCTL, style=customc]
/* This method is triggered whenever ioctl commands are issued from the user space */
ioctl_access(IOCTL_CMDS cmd) {	
	switch(cmd) {
		case CTXT_SWITCH: 
			req_ctxt_switch(thread_id);//requests for context switch
			break;
		case SIGNAL_OTHER-THREADS:
			Increment_curr_clk(thread_id); //this will increment the current clk for the given thread id.
			signal_all_other_threads(thread_id);
			break;
		case GET_CURR_CLK:
			get_curr_clk();//returns the current vector clock.
			break;
		case RESET_CLK:
			reset_clk(); //reset the current vector clock to zero.
			break:		
		case SET_CURR_CLK:
			set_curr_clk(clk);//sets the current vector clock with the clk received.
	}
}

//Methods of interest with respect to the ioctl cmds
mem_access check_mem_access_with_trace(thread_id_t tid) {
	...
	//method internally calls check_mem_acc_perm() with current clock time and uses the first valid instance vector clock registered for a given thread in the trace array.
		
	//returns e_ma_allowed|e_ma_restricted based on the check_mem_acc_perm()
}

ctxt_switch_thread(thread_id_t tid) {	
	down(threads_sem[tid-1]); //perform semaphore down operation respective semaphore.
	/**if the value is already 0 when performing the down, the thread waits until the value is positive.*/
}

signal_all_other_threads(thread_id_t tid) {
	//critical section for wait queue
	for i in(0,THREAD_COUNT) {
		if(i!=(tid-1)) {
			if(check_mem_access_with_trace(i+1) == e_ma_allowed) {
				/**Performs up operation on the respective thread semaphore.*/
				up(threads_sem[i]);
				wait_queue[i]=0;			
			}		
		}
	}	
	
	//critical section ends.
}

req_ctxt_switch(thread_id_t tid) {
		if(check_mem_access_with_trace(tid) == e_ma_restricted) {

		signal_all_other_threads(tid);
		
		//critical section for waitqueue
		wait_queue[tid-1] = 1; //sets the thread inline for waiting
		//critical section ends.
		
		ctxt_switch_thread(tid);

	}
}

\end{lstlisting}

\subsection{Design with an additional scheduler thread}\label{sec_add_thread}

In this design, we create an additional scheduler thread primarily addressing the signaling mechanism pertained in the previous design. 
By having an additional scheduler thread, we move the entire signaling system to the scheduler thread.
Thus, reducing the execution overhead encountered in the user space thread for signaling other threads.

The major change from the previous design apart from additional thread is in the memory assessment block.

\subsubsection*{Memory assessment block}
\begin{tikzpicture}[scale=0.7,transform shape]
 
 
  % memory assessment area
  
  \path \spacelayer {BEFOREMA}{Before M.A}{A callback is triggered before memory access is made to the global memory};
  \path (BEFOREMA.south)+(0.0,-2.0) \spacelayer {REQCTXTSWITCH}{Request \\ Context Switch}{ioctl(CTXT\_SWITCH, tid)};
  
  \path (REQCTXTSWITCH.south)+(0.0,-2.0) \spacelayer {MEMACCESS}{Memory\\ Access}{The Actual global Memory Access by the thread};

  \path (MEMACCESS.south)+(0.0,-2.0) \spacelayer {AFTERMA}{After M.A}{A callback is triggered after memory access is made to the global memory};
  \path (AFTERMA.south)+(0.0,-3.0) \spacelayer {SETCLK}{Setting \\ Global Vector clock}{ioctl(SET\_CLK, tid)};  
  
  
  \path (REQCTXTSWITCH.east)+(9.0,2.0) \spacelayer {CONTEXTSWITCH}{Context Switch call}{If memory access is restricted, signal\_other\_threads() and call down(thread\_sem[tid]) }; 
  \path (CONTEXTSWITCH.east)+(3.0,-5.0) \spacelayer {CHECKTRACE}{Check trace}{Checks if the execution in the input trace is valid or not.}; 
  \path (CONTEXTSWITCH.west)+(-2.0,-6.0) \spacelayer {THREADSEM}{THREAD\\ Semaphore}{Array of semaphores used by corresponding thread. Up and Down calls are made.}; 
  \path (CONTEXTSWITCH.south)+(0.0,-10.0) \spacelayer {SIGNALOTHERS}{Signal other\\ Threads}{Signals all the permitted threads to be resumed by calling up() on their respective semaphores. For verifying the permission internal call is made to checktrace}; 
 \path (SIGNALOTHERS.west)+(-3.0,0.0) \spacelayer {SETVECCLK}{Set Vector\\ Clock}{Kernel adaptation of set vector clock method invoked on ioctl call.}; 
 \path (SIGNALOTHERS.east)+(+3.0,0.0) \spacelayer {ADDTHREAD}{Additional\\ Thread}{Additional thread invokes signalothers method for every x ms.};  
  

  % Draw arrows between elements

 
 
  

  %memory assessment block
  \path [line] (BEFOREMA.south) -- node [left] {$req$} (REQCTXTSWITCH);
  \path [line] (AFTERMA.south) -- node [left] {$req$} (SETCLK);

  \path [line] (REQCTXTSWITCH.east) -- node [midway,above,sloped] {$ioctl()$} (CONTEXTSWITCH);
  
  \path [line] (CONTEXTSWITCH.south) -- node [left] {} (CHECKTRACE);
  \path [line] (CONTEXTSWITCH.south) -- node [midway,above,sloped] {down(sem[tid])} (THREADSEM);
  \path [line] (CONTEXTSWITCH.south) -- node [left] {} (SIGNALOTHERS);
  \path [line] (CHECKTRACE.north) -- node [midway,above,sloped] {allowed\ or\ restricted} (CONTEXTSWITCH);  
   
  \path [line] (SIGNALOTHERS.north) -- node [left] {} (CHECKTRACE);   
  \path [line] (CHECKTRACE.south) -- node [midway,above,sloped] {allowed\ or\ restricted} (SIGNALOTHERS);
  \path [line] (SIGNALOTHERS.north) -- node [midway,above,sloped] {up(sem[tid])} (THREADSEM);
   
   \path [line] (SETCLK.east) -- node [midway,above,sloped] {$ioctl()$} (SETVECCLK);
  %\path [linepart] (CONTEXTSWITCH.west) -- node [above] {$thread$}
                                 %node [right]{$resume$} (THREADRESUME);
   \path [line] (ADDTHREAD.west) -- node [left] {} (SIGNALOTHERS);
  %background generation block
 
  \background{BEFOREMA}{BEFOREMA}{SETCLK}{SETCLK}{User Space}
  \background{SETVECCLK}{CONTEXTSWITCH}{CHECKTRACE}{SIGNALOTHERS}{Kernel Space}
  

\end{tikzpicture}
\newpage
\subsubsection*{Pseudo Implementation}

The major changes are in kernel space code. 
However, there are minor variations in the \emph{AfterMA()} in user space. 

\begin{lstlisting}[title=User Space Implementation, style=customc]
//Rest of the code remains the same

AfterMA() {	
	ioctl(SET_MY_CLK, thread_id);
}

//Rest of the code remains the same

\end{lstlisting}

\begin{lstlisting}[title=Kernel Space - General module definitions, style=customc]
//code remains the same

signal_permitted_threads() {
	//critical section for wait queue
	for i in(0,THREAD_COUNT) {
		if(wait_queue[i] == 1) {
			if(check_mem_access_with_trace(i+1) == e_ma_allowed) {
				/**Performs up operation on the respective thread semaphore.*/
				up(threads_sem[i]);
				wait_queue[i]=0;			
			}		
		}
	}	
	
	//critical section ends.
}

module_init() {
	//code remains the same
	
	kernel_thread tk = create_kernel_thread(signal_permitted_threads)
	tk->setTimerCallForEvery(x) //this method will make call to signal permitted threads for every x ms.
}

//code remains the same
\end{lstlisting}


\section{Design with checking in user space}

In the following designs, we address the use of check permission of memory access method both in User Space and Kernel space.

\subsection{Design with no additional scheduler thread}

Without an additional thread in kernel space, the design would require a signaling function inside \emph{AfterMA()}, similar to the one used in Design~\ref{no_check_no_add}. 
Triggering a signaling mechanism is an additional overhead on the thread calling the \emph{AfterMA()}. 
Therefore, such a design is not a wise choice when considering the performance metrics such as execution time.
 
\subsection{Design with an additional scheduler thread}

The scheduler implementation is similar to one defined in the section \ref{sec_add_thread}. 
Key difference is the additional checking for memory permissions in the user space. 

\subsubsection*{Memory assessment block}
\begin{tikzpicture}[scale=0.7,transform shape]
 
 
  % memory assessment area

  \path \spacelayer {BEFOREMA}{Before M.A}{A callback is triggered before memory access is made to the global memory. If memory access is restricted, req\_ctxt()};
  \path (BEFOREMA.south)+(-3.0,-2.5) \spacelayer {CHECKTRACEUSR}{Check trace}{Checks if the execution in the input trace is valid or not.};   
  
  \path (BEFOREMA.south)+(0.0,-6.0) \spacelayer {REQCTXTSWITCH}{Request \\ Context Switch}{ioctl(CTXT\_SWITCH, tid)};
  
  \path (REQCTXTSWITCH.south)+(-3.0,-2.0) \spacelayer {MEMACCESS}{Memory\\ Access}{The Actual global Memory Access by the thread};

  \path (MEMACCESS.south)+(3.0,-2.0) \spacelayer {AFTERMA}{After M.A}{A callback is triggered after memory access is made to the global memory};
  \path (AFTERMA.south)+(0.0,-3.0) \spacelayer {SETCLK}{Setting \\ Global Vector clock}{ioctl(SET\_CLK, tid)};  
  
  
  \path (REQCTXTSWITCH.east)+(9.0,2.0) \spacelayer {CONTEXTSWITCH}{Context Switch call}{If memory access is restricted, signal\_other\_threads() and call down(thread\_sem[tid]) }; 
  \path (CONTEXTSWITCH.east)+(3.0,-5.0) \spacelayer {CHECKTRACE}{Check trace}{Checks if the execution in the input trace is valid or not.}; 
  \path (CONTEXTSWITCH.west)+(-2.0,-6.0) \spacelayer {THREADSEM}{THREAD\\ Semaphore}{Array of semaphores used by corresponding thread. Up and Down calls are made.}; 
  \path (CONTEXTSWITCH.south)+(0.0,-10.0) \spacelayer {SIGNALOTHERS}{Signal other\\ Threads}{Signals all the permitted threads to be resumed by calling up() on their respective semaphores. For verifying the permission internal call is made to checktrace}; 
 \path (SIGNALOTHERS.west)+(-3.0,0.0) \spacelayer {SETVECCLK}{Set Vector\\ Clock}{Kernel adaptation of set vector clock method invoked on ioctl call.}; 
 \path (SIGNALOTHERS.east)+(+3.0,0.0) \spacelayer {ADDTHREAD}{Additional\\ Thread}{Additional thread invokes signalothers method for every x ms.};  
  

  % Draw arrows between elements

 
 
  

  %memory assessment block
  \path [line] (BEFOREMA.south) -- node [left] {} (CHECKTRACEUSR);
  \path [line] (CHECKTRACEUSR.south) -- node  [midway,right] {$restricted$} (REQCTXTSWITCH);
  \path [line] (CHECKTRACEUSR.south) -- node  [left] {$allowed$} (MEMACCESS);
  \path [line] (AFTERMA.south) -- node [left] {$req$} (SETCLK);

  \path [line] (REQCTXTSWITCH.east) -- node [midway,above,sloped] {$ioctl()$} (CONTEXTSWITCH);
  
  \path [line] (CONTEXTSWITCH.south) -- node [left] {} (CHECKTRACE);
  \path [line] (CONTEXTSWITCH.south) -- node [midway,above,sloped] {down(sem[tid])} (THREADSEM);
  \path [line] (CONTEXTSWITCH.south) -- node [left] {} (SIGNALOTHERS);
  \path [line] (CHECKTRACE.north) -- node [midway,above,sloped] {allowed\ or\ restricted} (CONTEXTSWITCH);  
   
  \path [line] (SIGNALOTHERS.north) -- node [left] {} (CHECKTRACE);   
  \path [line] (CHECKTRACE.south) -- node [midway,above,sloped] {allowed\ or\ restricted} (SIGNALOTHERS);
  \path [line] (SIGNALOTHERS.north) -- node [midway,above,sloped] {up(sem[tid])} (THREADSEM);
   
   \path [line] (SETCLK.east) -- node [midway,above,sloped] {$ioctl()$} (SETVECCLK);
  %\path [linepart] (CONTEXTSWITCH.west) -- node [above] {$thread$}
                                 %node [right]{$resume$} (THREADRESUME);
   \path [line] (ADDTHREAD.west) -- node [left] {} (SIGNALOTHERS);
  %background generation block
 
  \background{CHECKTRACEUSR}{BEFOREMA}{SETCLK}{SETCLK}{User Space}
  \background{SETVECCLK}{CONTEXTSWITCH}{CHECKTRACE}{SIGNALOTHERS}{Kernel Space}
  

\end{tikzpicture}
\subsubsection*{Pseudo Implementation}

The major changes are in user space code. 


\begin{lstlisting}[title=User Space Implementation, style=customc]
//Rest of the code remains the same


mem_access ma_status[THREAD_COUNT];
vec_clk curr_clk_time;

initialize_vec_clock() {

    for i in range(0, THREAD_COUNT)
    {
        curr_clk_time.clocks[i] = 0;
    }
}

BeforeMA() {

	ma_status[thread_id-1] = check_mem_access_with_trace(thread_id);
	if(ma_status[id-1] == e_ma_restricted) {
		ioctl(CTXT_SWITCH, thread_id);	
	}
	  
}

AfterMA() {	
	ioctl(SET_MY_CLK, thread_id);
	curr_clk_time.clocks[thread_id-1]++;
}

//Rest of the code remains the same

\end{lstlisting}

\section{Variant in blocking implementation}

In the previous designs, the blocking was done using semaphores. 
In the variant design, we use the combination of \emph{schedule()} and \emph{wake\_up\_process()} functions provided by the Linux scheduler APIs. 
The kernel level tasks associated for the provided user level threads are moved from running queue to wait queue by initially setting the task status as \emph{TASK\_INTERRUPTIBLE} and yielding the processor by invoking \emph{schedule()}. 
The task added in wait queue is later resumed, when \emph{wake\_up\_process(sleeping\_task)} is invoked by another task. 
On calling the \emph{wake\_up\_process(sleeping\_task)}, the task status for \emph{sleeping\_task} is set as \emph{TASK\_RUNNING}. 
It would be pushed to run queue and executed in future by the operating system scheduler on the basis of scheduler class and priority of tasks in run queue. 
 
 
\subsubsection*{Variant Pseudo Code for Design \ref{no_check_no_add}}

\begin{lstlisting}[title=Kernel Space - General module definitions, style=customc]
//code remains the same.
typedef struct {
	int is_waiting;
	struct task_struct *my_task;
}wait_queue_threads_t;


static wait_queue_threads_t wait_queue[THREAD_COUNT];

module_init() {
	for i in range(0,THREAD_COUNT) {
		wait_queue[i].is_waiting = 0; 
		wait_queue[i].my_task = NULL;
		curr_clk[i] = 0;
	}
	alloc_ioctl_device();//method used to allocate ioctl device.
}

//code remains the same
\end{lstlisting}

\begin{lstlisting}[title=Kernel Space - IOCTL, style=customc]
//code remains the same

ctxt_switch_thread(thread_id_t tid) {	
	//critical section for wait queue
	wait_queue[tid-1].is_waiting = 1;
	wait_queue[tid-1].my_task = current;
	set_current_state(TASK_INTERRUPTIBLE);
	//critical section ends
	schedule();
}

signal_all_other_threads(thread_id_t tid) {
	//critical section for wait queue
	for i in(0,THREAD_COUNT) {
		if(i!=(tid-1)&&(wait_queue[i].is_waiting==1)) {
			if(check_mem_access_with_trace(i+1) == e_ma_allowed) {
				wait_queue[i].is_waiting = 0;
				wake_up_process(wait_queue[i].my_task);				
			}		
		}
	}	
	
	//critical section ends.
}


\end{lstlisting}

\subsubsection*{Variant Pseudo Code for Design \ref{sec_add_thread}}

\begin{lstlisting}[title=Kernel Space - General module definitions, style=customc]
//code remains the same.
typedef struct {
	int is_waiting;
	struct task_struct *my_task;
}wait_queue_threads_t;


static wait_queue_threads_t wait_queue[THREAD_COUNT];

module_init() {
	for i in range(0,THREAD_COUNT) {
		wait_queue[i].is_waiting = 0; 
		wait_queue[i].my_task = NULL;
		curr_clk[i] = 0;
	}
	alloc_ioctl_device();//method used to allocate ioctl device.
}

//code remains the same
\end{lstlisting}

\begin{lstlisting}[title=Kernel Space - General module definitions, style=customc]
//code remains the same

signal_permitted_threads() {
	//critical section for wait queue
	for i in(0,THREAD_COUNT) {
		if(wait_queue[i].is_waiting==1) {
			if(check_mem_access_with_trace(i+1) == e_ma_allowed) {
				/**Performs up operation on the respective thread semaphore.*/
				wait_queue[i].is_waiting = 0;
				wake_up_process(wait_queue[i].my_task);	
			}		
		}
	}	
	
	//critical section ends.
}

//code remains the same
\end{lstlisting}

	%\input{content2}
	
	%======================================================
	% The back matter
	%======================================================
	%\cleardoublepage
	\refstepcounter{dummy}
	\addcontentsline{toc}{chapter}{\bibname}
	\bibliographystyle{plainnat} % <--- layout of the bib
	\bibliography{bibl_2} % file name of your bib

\end{document}
%======================================================
%======================================================
