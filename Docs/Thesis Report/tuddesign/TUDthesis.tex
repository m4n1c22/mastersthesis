\documentclass[type=msc,nochapterpage,colorback,accentcolor=tud9c]{tudthesis}

\newcommand{\getmydate}{%
  \ifcase\month%
    \or Januar\or Februar\or M\"arz%
    \or April\or Mai\or Juni\or Juli%
    \or August\or September\or Oktober%
    \or November\or Dezember%
  \fi\ \number\year%
}

\begin{document}
  \thesistitle{Realizing Iterative Relaxed Scheduler in Kernel Space}%
    {}
  \author{Sreeram Sadasivam}
  \referee{Patrick Metzler}{Prof Dr Neeraj Suri}
  \department{Fachbereich Informatik}
  \group{DEEDS}
%  \dateofexam{\today}{\today}
  \makethesistitle
  \affidavit{Sreeram Sadasivam}

  \chapter{Generelle Informationen}
    Die Klasse basiert auf der \textaccent{tudreport}-Klasse von C. v. Loewenich und
    J. Werner. Alle "Anderungen dort wirken sich direkt auf die
    \textaccent{tudthesis}-Klasse aus. Genauer: die \textaccent{tudthesis}-Klasse definiert nur einige
    neue Befehle und legt die Formatierung der ersten zwei Seiten (Titelseite
    und R"uckseite des Titleblattes) fest.

\chapter{Verwendung der Klasse}
  Die Klasse wird verwendet, indem in der Dokumentenpr"aambel
  \textaccent{\textbackslash documentclass\{tudthesis\}}
  eingetragen wird.

  \section{Klassenoptionen}
    Die Klasse unterst"utzt alle Klassenoptionen der tudreport-Klasse.
    \paragraph{Neue Optionen}
    \begin{itemize}
      \item \textbf{type=<dr|drfinal|diplom|msc|pp|bsc|sta>}\\
        Hiermit wird die Art der Arbeit angegeben. Dies legt verschiedene
        Formatierungen fest.\\
        \begin{tabular}{ll}
        \textaccent{dr} &f"ur eingereichte Dissertationen\\
        \textaccent{drfinal} &f"ur genehmigte Dissertationen\\
        \textaccent{diplom} &f"ur Diplomarbeiten\\
        \textaccent{msc} &f"ur Master-Theses\\
        \textaccent{pp} &f"ur Project-Proposals\\
        \textaccent{bsc} &f"ur Bachelor-Theses\\
        \textaccent{sta} &f"ur Studienarbeiten
        \end{tabular}
    \end{itemize}

  \section{Befehle}
    \begin{itemize}\itemsep-0.5\parsep
      \item \textbf{\textbackslash thesistitle\{\#1\}\{\#2\}}\\
        \#1: Titel der Arbeit in der Erstsprache (z.B. Deutsch)\\
        \#2: Titel der Arbeit in der zweiten Sprache (z.B. Englisch)
      \item \textbf{\textbackslash makethesistitle}\\
        Erzeugt die korrekte Titelseite
      \item \textbf{\textbackslash author\{\#1\}}\\
        \#1: Name des Autors, bei Dr.-Arbeit zus"atzlich auch Geburtsort!
      \item \textbf{\textbackslash date\{\#1\}}\\
        Standard ist der aktuelle Monat und das aktuelle Jahr (z.B. \getmydate)\\
        \#1: individuelles Datum
      \item \textbf{\textbackslash referee\{\#1\}\{\#2\}[\#3]}\\
        Namen der Gutachter, (3. Gutachter optional)
      \item \textbf{\textbackslash department\{\#1\}}\\
        Fachbereich an dem die Arbeit durchgef"uhrt wurde. Standard ist
        \glqq Fachbereich Physik\grqq.
      \item \textbf{\textbackslash group\{\#1\}}\\
        Arbeitsgruppe / Institut an dem die Arbeit durchgef"uhrt wurde
      \item \textbf{\textbackslash dateofexam\{\#1\}\{\#2\}}\\
        \#1: Tag der Einreichung der Arbeit\\
        \#2: Tag der Pr"ufung / Tag des Abschlusses\\
        \textaccentcolor{Wird nur bei \textaccent{type=drfinal} verwendet.
        Ansonsten wird ein leeres Feld erzeugt, in das bei Abgabe der 
        Arbeit ein Stempel gesetzt wird.}
      \item  \textbf{\textbackslash affidavit[\#1]\{\#2\}}\\
        \#1: Datum der Eigenst"andigkeitserkl"arung (optional)\\
        \#2: Signatur unter der Unterschrift\\
    \end{itemize}

\end{document}
