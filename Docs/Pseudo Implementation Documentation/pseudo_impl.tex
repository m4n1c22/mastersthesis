% Author: Sreeram Sadasivam
%\documentclass{minimal}
%
%\usepackage{pgf}
%\usepackage{tikz}
%\usepackage[utf8]{inputenc}
%\usetikzlibrary{arrows,automata}
%\usetikzlibrary{positioning}
%\usetikzlibrary{shadows,arrows}
%% Define the layers to draw the diagram
%\pgfdeclarelayer{background}
%\pgfdeclarelayer{foreground}
%\pgfsetlayers{background,main,foreground}
% 
%
%\tikzset{
%    state/.style={
%           rectangle,
%           rounded corners,
%           draw=black, very thick,
%           minimum height=2em,
%           inner sep=2pt,
%           text centered,
%           },
%}
%
%\begin{document}
%\tikzstyle{line} = [draw, -latex']
%\begin{tikzpicture}[->,>=stealth']
%
% % Position of THREAD_REG 
% % Use previously defined 'state' as layout (see above)
% % use tabular for content to get columns/rows
% % parbox to limit width of the listing
% \node[state] (THREAD_REG) 
% {\begin{tabular}{l}
%  \textbf{Thread Registration}\\
%  \parbox{4cm}{\begin{itemize}
%   \item Start of user program
%   \item register\_threads()
%   \item Parameter $thread\_IDs[thread\_count]$
%  \end{itemize}
%  }%\\[4em]
%
% \end{tabular}};
%  
% % State: PROCFS with different content
% \node[state,    	% layout (defined above)
%  text width=3cm, 	% max text width
%  %yshift=2cm, 		% move 2cm in y
%  right of=THREAD_REG, 	% Position is to the right of THREAD_REG
%  node distance=6cm, 	% distance to THREAD_REG
%  anchor=center] (PROCFS) 	% posistion relative to the center of the 'box'
% {%
% \begin{tabular}{l} 	% content
%  \textbf{Registration File}\\
%  \parbox{2.8cm}{Proc FS custom file for communication}
% \end{tabular}
% };
% 
%
% % STATE SYSCALL
% \node[state,
%  right of=PROCFS,
%  node distance=5cm,
%  anchor=center] (SYSCALL) 
% {%
% \begin{tabular}{l}
%  \textbf{SYSCALL}\\
%  \parbox{3cm}{write() system call }
% \end{tabular}
% };
% 
% 
% % STATE CALLBACK
% \node[state,
%  below of=SYSCALL,
%  yshift=-2cm,
%  anchor=center,
%  text width=3cm] (CALLBACK) 
% {%
% \begin{tabular}{l}
%  \textbf{Write Callback}\\
%  \parbox{2.8cm}{Callback function triggered inside LKM during write}
% \end{tabular}
% };
%
% % draw the paths and and print some Text below/above the graph
% \path (THREAD_REG) 	edge  node[anchor=south,above]{$filewrite()$} (PROCFS)                                   
% 		(PROCFS)       	edge  node[anchor=south,above]{$write()$} 	(SYSCALL);
% 
% \path [line,dashed] (SYSCALL) -- node{$callback\ triggered$}(CALLBACK);
%
%
%
%\end{tikzpicture}
%
%
%\end{document}
\documentclass[11pt]{article}
\usepackage{tikz}
\usetikzlibrary{shadows,arrows}
% Define the layers to draw the diagram
\pgfdeclarelayer{background}
\pgfdeclarelayer{foreground}
\pgfsetlayers{background,main,foreground}
 
% Define block styles  
\tikzstyle{materia}=[draw, fill=blue!20, text width=7.0em, text centered,
  minimum height=1.5em,drop shadow]
\tikzstyle{practica} = [materia, text width=8em, minimum width=10em,
  minimum height=3em, rounded corners, drop shadow]
\tikzstyle{texto} = [above, text width=6em, text centered]
\tikzstyle{linepart} = [draw, thick, color=black!50, -latex', dashed]
\tikzstyle{line} = [draw, thick, color=black!50, -latex']
\tikzstyle{ur}=[draw, text centered, minimum height=0.01em]
 
% Define distances for bordering
\newcommand{\blockdist}{1.3}
\newcommand{\edgedist}{1.5}

\newcommand{\practica}[2]{node (p#1) [practica]
  {Pr\'actica #1\\{\scriptsize\textit{#2}}}}


\newcommand{\spacelayer}[3]{node (p#1) [practica]
  {#2\\{\scriptsize\textit{#3}}}}


% Draw background
\newcommand{\background}[5]{%
  \begin{pgfonlayer}{background}
    % Left-top corner of the background rectangle
    \path (#1.west |- #2.north)+(-0.5,0.5) node (a1) {};
    % Right-bottom corner of the background rectanle
    \path (#3.east |- #4.south)+(+0.5,-0.25) node (a2) {};
    % Draw the background
    \path[fill=yellow!20,rounded corners, draw=black!50, dashed]
      (a1) rectangle (a2);
    \path (a1.east |- a1.south)+(0.8,-0.3) node (u1)[texto]
      {\scriptsize\textit{#5}};
  \end{pgfonlayer}}

\newcommand{\transreceptor}[3]{%
  \path [linepart] (#1.east) -- node [above]
    {\scriptsize Transreceptor #2} (#3);}

\begin{document}
\begin{tikzpicture}[scale=0.7,transform shape]
 
  % Draw diagram elements
  % thread registration block area
  \path \spacelayer {1}{Thread\\ Registration}{reg\_threads()};  
  \path (p1.east)+(5.0,0.0) \spacelayer{2}{Registration File}{Proc FS custom file for communication};
  \path (p2.east)+(4.5,0.0) \spacelayer{3}{SYSCALL}{write() system call};
  \path (p3.south)+(0.0,-2.0) \spacelayer{4}{Write Callback}{Callback function triggered inside LKM during write};

  % memory assessment area
  \path (p1.south)+(0.0,-5.0) \spacelayer {5}{Before M.A}{A callback is triggered before memory access is made to the global memory};
  \path (p5.south)+(0.0,-2.0) \spacelayer {6}{Memory\\ Assessment}{Checks if the memory access is permitted or not};
  \path (p6.south)+(0.0,-2.0) \spacelayer {7}{Scheduler\\ yield}{scheduler\_yield()};

  % Draw arrows between elements
  \path [line] (p1.east) -- node [above] {$filewrite()$} (p2);

  \path [line] (p2.east) -- node [above] {$write()$} (p3);
  
  \path [linepart] (p3.south) -- node [left] {$callback$}
                                 node [right]{$triggered$} (p4);


  \path [line] (p5.south) -- node [left] {$Check$}
						     node [right] {$Trace$} (p6);

  \path [line] (p6.south) -- node [left] {$Not$}
						     node [right] {$Permitted$} (p7);

  \background{p1}{p1}{p7}{p7}{User Space}
  \background{p3}{p3}{p4}{p4}{Kernel Space}


\end{tikzpicture}
\end{document} 