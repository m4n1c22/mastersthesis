\documentclass[12pt]{article}
\usepackage{graphicx} 
\usepackage{verbatim}
\usepackage{todonotes}
%\usepackage{hyperref}
\usepackage{comment}
\newcommand{\TODO}[1]{\todo[inline]{#1}}

\begin{document}
\title{\vspace{-3.5cm} Thesis Proposal}


\author{
		 Sreeram Sadasivam\\
		M.Sc Distributed Software Systems\\
		TU Darmstadt, Germany\\
		sreeram.sadasivam@stud.tu-darmstadt.de
}

\date{}


\maketitle




\section*{Overview}

Concurrency bugs which are often resident in multi-threaded programs with shared memory designs are difficulty to find and reproduce. 
Deterministic multi-threading(DMT) is one such scheme indicated to resolve the above difficulty. 
But, DMT presents the challenge of having no scheduling constraints.  
However, currently there are no such techniques that allow to control the schedule of a multi-threaded program on a fine-grained level, i.e, on the level of single memory accesses. 
A design with a granularity of single memory accesses would help in enforcing the scheduling constraint. 

Consider an example, there are two memory accesses $MA_{1}$ and $MA_{2}$ from two user level threads $T{1}$ and $T{2}$ respectively. 
The generated constraint for execution is $MA_{1}$ after $MA_{2}$. 
Let us assume, $T{1}$ reaches $MA_{1}$ before $T{2}$ executes $MA_{2}$. 
The scheduling design has to detect the memory access for $MA_{2}$ and $MA_{1}$ for enforcing the above constraint. 
With detection in place, the scheduler would enforce a conditional wait or busy wait on $T{1}$ until $T{2}$ completes its execution of $MA_{2}$. 
Moving the scheduling decision to kernel space would remove the above mentioned synchronization required for scheduling. 
Thus, improving the execution time of the user program.


In the existing design, we have a thread scheduler and an verification engine.  
The verification engine primarily focusses on instrumenting the user code and realizing memory accesses made by various user threads. 
The set of safe schedules are provided by the verification engine for the given user program. 
The generated execution pattern is later realized with the thread scheduler, when the user program is executed. 
The scheduler thread is realized in user space. 
It is realized in two design modes - Seperate thread and shared thread. 
Context switching of the scheduler thread is possible when executed in user space. 
There is also a possibility for the operating system scheduler to override the scheduler constraints set by the scheduler thread on its user threads. 
Moving the scheduler task to the kernel space would help to enforce the scheduling constraints set by the user. 
The proposed design can be realized by having the thread scheduler as a Loadable Kernel Module. 


Such a design would be benchmarked on various thread conditional scenarios such reader-writer problems, Peterson solution, Lamport solution. 
These programs enforce the verification of correctness in multi-threaded environment. 
And also with Indexer and LastZero benchmarking programs. 
The evaluation is performed on the overhead exerted by the transition to the LKM module. 
And additional comparison of execution overhead generated by the rival designs PARROT, CORE-DET are also considered for the above mentioned benchmarking programs. 
The evaluation is scaled from the use of thread count pertaining to the core count, to the use of scaled up version of thread count overshadowing the core count. 
Thus, creating a possibility of false sharing situations and various other potential execution overhead conditions. 

\end{document}
